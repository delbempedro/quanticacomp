\documentclass[12pt, a4paper]{article} %determina o tamanho da fonte, o tipo de papel e o tipo de documento.

\setlength{\parindent}{1.0 cm} %tamanho do espaço para começar o parágrafo.
\setlength{\parskip}{0.5cm} %tamanho do espaço entre os parágrafos.

%Aqui ficam os pacotes utilizados para formatação do documento de modo geral:

\usepackage[utf8]{inputenc} 
\usepackage{indentfirst} %Coloca espaços nos inícios de parágrafos automaticamente. 
\usepackage[brazilian]{babel} %
\usepackage{amsmath}
\usepackage[hmargin=3cm, vmargin=2.5cm, bmargin=2.5cm]{geometry}
\usepackage{multicol}
\usepackage{graphicx} %para poder inserir imagens
\usepackage{subfig}
\usepackage{booktabs} 
\usepackage{hyperref} %para poder adicionar links e hiperlinks
\usepackage{float} %para poder posicionar as imagens


\usepackage{listings} %para poder incluir códigos
\usepackage{xcolor}
\definecolor{codegreen}{rgb}{0,0.6,0}
\definecolor{codegray}{rgb}{0.5,0.5,0.5}
\definecolor{codepurple}{rgb}{0.58,0,0.82}
\definecolor{backcolour}{rgb}{0.95,0.95,0.92}
\lstdefinestyle{mystyle}{
    backgroundcolor=\color{backcolour},   
    commentstyle=\color{codegreen},
    keywordstyle=\color{magenta},
    numberstyle=\tiny\color{codegray},
    stringstyle=\color{codepurple},
    basicstyle=\ttfamily\footnotesize,
    breakatwhitespace=false,         
    breaklines=true,                 
    captionpos=b,                    
    keepspaces=true,                 
    numbers=left,                    
    numbersep=5pt,                  
    showspaces=false,                
    showstringspaces=false,
    showtabs=false,                  
    tabsize=2,
    morecomment={l}[!],
    language=[77]Fortran,
}
\lstset{style=mystyle}

\begin{document} %começa alguma coisa,neste caso, o documento, sempre importante lembrar de colocar o \end{} para não dar erro 
	
	\begin{titlepage}
		\begin{center}
\Huge{Universidade de São Paulo}\\
\large{Instituto de Física de São Carlos}\\
\vspace{20pt}
\vspace{200pt}
\textbf{Lista 3}\\
\vspace{8cm}
		\end{center}

\begin{flushleft}
\begin{tabbing}
Pedro Calligaris Delbem 5255417\\
\end{tabbing}
\vspace{0.5cm}
Professor: Attilio Cucchieri\\		
		\end{flushleft}
	
		\begin{center}
			\vspace{\fill}
	Abril de 2025	
		\end{center}
	\end{titlepage}

%####################################################################### SUMÁRIO
	\tableofcontents 
	\thispagestyle{empty}
	\newpage
%#########################################################################

\section{The Numerov Algorithm}

    \subsection{Exerc\'icio 1}

        Tarefa: Reolver a equa\c{c}\~ao de Poisson para $\hat{\phi}(r)$ definido por $\frac{\hat{\phi}(r)}{r} := \phi(r)$ onde $\phi(r)$ \'e o potencial eletrost\'atico e a densidade de carga \'e $\rho(r) = \frac{e^{-r}}{8 \pi}$, considerando simetria esf\'erica.
        
        Deve-se resolver das seguintes maneiras:
        \begin{itemize} 
            \item Pelo algoritmo de Numerov
            \begin{itemize}
                \item Escolhendo $\hat{\phi}(0)$ e $\hat{\phi}(\delta r)$, para $r \approx 0$
                \item Escolhendo $\hat{\phi}(0)$ e $\hat{\phi}(\delta r)$, r muito grande (o equivalente num\'erico a $r \to \infty$)
            \end{itemize}
            \item Analiticamente
        \end{itemize}

        Primeiro deve-se manipular a equa\c{c}\~ao de Poisson de modo a obter uma equa\c{c}\~ao para $\hat{\phi}(r)$

        A equa\c{c}\~ao de Poisson \'e:
        \begin{equation*}
            \nabla^{2} \phi(r) = -4\pi \rho(r)
        \end{equation*}

        Sabemos que $\nabla^{2}$ em coordenadas esf\'ericas \'e:
        \begin{equation*}
            \nabla^{2} = \frac{1}{r^{2}} \left( \frac{\partial}{\partial r} \left( r^{2} \frac{\partial}{\partial r} \right) + \frac{1}{\sin \theta}\frac{\partial}{\partial \theta} \left( \sin \theta \frac{\partial}{\partial \theta} \right) + \frac{1}{r \sin^{2} \theta} \frac{\partial^{2}}{\partial \phi^{2}} \right)
        \end{equation*}

        Pela simetria radial reduzimos  para:

        \begin{equation*}
            \nabla^{2} = \frac{1}{r^{2}} \frac{\partial}{\partial r} \left( r^{2} \frac{\partial}{\partial r} \right)
        \end{equation*}

        Substituimos $\frac{\hat{\phi}(r)}{r} := \phi(r)$:

        \begin{equation*}
            \nabla^{2} \phi(r) = \frac{1}{r^{2}} \frac{\partial}{\partial r} \left( r^{2} \frac{\partial}{\partial r} \left( \frac{\hat{\phi}(r)}{r} \right) \right)
        \end{equation*}

        Aplicando as derivadas:

        \begin{equation*}
            \nabla^{2} \phi(r) = \frac{1}{r}\frac{\partial^{2}}{\partial r^{2}} \hat{\phi}(r)
        \end{equation*}

        Substituindo na equa\c{c}\~ao de Poisson:

        \begin{equation*}
            \frac{1}{r} \frac{d^{2} \hat{\phi}(r)}{dr^{2}} = -4\pi \rho(r) = -\frac{e^{-r}}{2}
        \end{equation*}

        Assim, obtemos:

        \begin{equation*}
            \frac{d^{2} \hat{\phi}(r)}{dr^{2}} = -\frac{re^{-r}}{2}
        \end{equation*}



        \subsubsection{Resolu\c{c}\~ao Anal\'itica}

            Integrando a equa\c{c}\~ao, com rela\c{c}\~ao ao r duas vezes, obtemos:

            \begin{equation*}
                \hat{\phi}(r) = e^{-r}\left(1 + \frac{r}{2}\right) + C_{1}r + C_{2}
            \end{equation*}

            Toma-se $\hat{\phi}(0)$ = -1 e $\hat{\phi}(\infty)$ = 0, obtemos:

            \begin{equation*}
                C_{1} = 0 \quad \text{e} \quad C_{2} = 0
            \end{equation*}
            
            E deste modo a solu\c{c}\~ao anal\'itica \'e:

            \begin{equation*}
                \hat{\phi}(r) = e^{-r}\left(1 + \frac{r}{2}\right)
            \end{equation*}

\end{document}